\subsection{Business Model}\label{sec:business-model}

There are various different definitions of a business model. Arguably the most popular one is as follows: \textit{``business model describes the rationale of how an organisation creates, delivers, and captures value}''~\cite{Osterwalder2010BusinessChallengers}. The business model is an important tool to conceptualise and record the relations and environment within which the organisation operates. Two approaches to business models are particularly interesting:
\begin{enumerate*}[label=(\roman*)] 
    \item The STOF model by Bouwmann et. al., which focuses on providing value for both customers and businesses by balancing the customers' requirements and strategic business interests~\cite{Bouwman2008ServiceModels}; and the
    \item \acrfull{bmc} by Osterwalder et. al. which puts more focus on simplicity and plug\&play approach for idea sharing and mapping~\cite{Hong2013CriticismsCanvas}.
\end{enumerate*}
 
In this synopsis, we further focus on the second one. Its main goal is to provide entrepreneurs, businessmen, investors and other people with a common language for discussing business model and innovation of a specific organisation. The canvas consists of 9 building blocks that are divided into four areas of an organisation: customers, offer, infrastructure and financial viability.

In the original definition of \acrshort{bmc}, the basic goal of an organisation is to make money~\cite[p. 21]{Osterwalder2010BusinessChallengers} and thus this canvas cannot be used by other types of organisations (e.g. NGOs, social enterprises). Although this has been criticised in the literature~\cite{Hong2013CriticismsCanvas}, we believe this does not prevent the \acrshort{bmc} to be widely adopted, nor does it prevent it to be applied on our scenario.

Further criticism of the \acrshort{bmc} include lack of consideration for competition or imbalance in the focus on different building blocks~\cite{Hong2013CriticismsCanvas}.

% TODO write about BMC applied to Azure/Microsoft/AI computing
\paragraph{Application of BMC to cloud-based machine learning services}
To better understand how the machine learning services can be provided as a cloud-based solution, we created a \acrshort{bmc}. We used a modified version of the original canvas called \textit{Advanced Business Model Canvas}, which adds sub-points into some of the boxes, making it easier to avoid ambiguity when filling the \acrshort{bmc}~\cite{King2017BusinessInstructions, Hong2013CriticismsCanvas}. The \acrshort{bmc} can be found in Appendix 1.
% TODO add reference to appendix

After filling the canvas using Microsoft's Azure Machine Learning Studio\footnotemark as an example of a service, we find that the business model used by Microsoft is sound and usable in the market. To our best knowledge, Microsoft does not currently offer consultancy services for the design of the \acrshort{ml} solution.
% 
\footnotetext{\url{https://studio.azureml.net/}, accessed 09-12-2018}
% 
Limited advice on how to achieve the best results, using Microsoft's solution is published with the documentation\footnotemark and best practice techniques can also be found in various online communities. The support provided by Microsoft, however is limited to how to use different parts of the tool and error handling. Large enterprise customers may also engage in \acrshort{sla} specifications. We find that the consultancy offering might further complement Microsoft's Machine Learning environment and bring additional source of revenue to the company.
% 
\footnotetext{\url{https://docs.microsoft.com/en-us/azure/machine-learning/studio/faq}, accessed 09-12-2018}

Based on this case, we also maintain that the sub-points are not an obstacle to the original purpose of the \acrshort{bmc}, as they do help with filling the canvas with relevant information. 