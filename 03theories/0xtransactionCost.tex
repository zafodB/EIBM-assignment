\subsection{Transaction cost} \label{sec:transaction-cost}
DOBRE TY - Tousind tak! ci jak sa to pise
Ale necital som to este :D
a ja som sa uz tesil ze dobre to je :D ale popravde este ani ja poriadne, rano to budem opravovat este

Another theory we look into is Transaction cost, which arise when an economic exchange occurs. It can occur between firms, but also within and it does not create any value directly. Ronald Coase argues that firms exists as a response to the high costs on the market, being it often cheaper to produce in-house than to create a contract with a supplier. Therefore, the objective of a firm is to reproduce the conditions of a competitive market for the factors of production within the firm at a lower cost than the actual market. How does the firm know then, when it is beneficial for it to do the production in-house and when it is better to contract the external firm? How does it know what processes should be performed in-house and what externally?

There are several costs identified with transaction cost:
\begin{itemize}[noitemsep]
    \item Search and information cost are ones that occur during a research whether the good needed is on the market and where is the lowest cost,
    \item Bargaining and decision cost are associated with agreement discussion with the supplier and creation of an appropriate contract,
    \item Policing and enforcement cost associated with supervising the contract and if needed taking legal actions if the contract is not followed.
\end{itemize}
As it can be seen, transaction costs are incurring while searching for the supplier, bargaining the final cost, creating the contract, monitoring the quality of goods and contract compliance and enforcing rules. Therefore, it is sometimes more efficient to produce the good in-house omitting the transaction cost even on the expense of higher production costs, than ‘’outsourcing’’ the production of the good. On top of that, the whole process is time-consuming which also costs money.

\paragraph{Effect of Transaction costs on cloud-based machine learning services}

Every service of Microsoft is running on cloud, from the customer’s point of view, but \acrshort{ms} itself needs to buy a lot of hardware in order to accommodate all services provided and satisfy the needs of customers. Because of that, it needs to contract suppliers for delivery of \acrshort{hw} needed, as \acrshort{ms} is not a producer of such highly specialized \acrshort{hw}, does not have a know how and a big investment would be need for \acrshort{ms} to produce given \acrshort{hw} in-house. Firm of this size is buying equipment on regular basis in bulk and in large quantities therefore, its bargaining power is great. As in \cite{Williamson1979Transaction-CostRelations} in Figure 7.1, the investment characteristic in this case is Mixed and frequency recurrent, as the \acrshort{ms} is purchasing customized material/equipment tailored to its needs.

