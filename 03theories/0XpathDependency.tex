\subsection{Path Dependency and Lock-in} \label{sec:path-depenency}
Path dependency is an economic theory, which explains how history or past decisions influences the present. The choices we made in the past influences the choices we make today. This holds for the continuous use of a standards, products or practices. Path dependence is caused by the behavior of doing things the way we know, out of uncertainty about the cost of alternatives, because the cost of conversion exceeds the immediate gains in operating efficiency. This way, we are locked in with suboptimal technology or inferior standard even though a more efficient one is available, as short-run benefits are often outweighs the long-run benefits. As Douglas North said: \textit{‘’Once a development path is set on a particular course, the network externalities, the learning process of organizations, and the historically derived modelling of the issues reinforces the course.’’}

There exist three degrees of path dependence as defined by \cite{Liebowitz2001WinnersTechnology}:
\begin{itemize}[noitemsep]
    \item First-degree Path Dependence is about decisions made in the past, based on correct predictions about the future however, the position in which the individual is may not seem optimal, though there is no error or inefficiency.
    \item In second-degree Path Dependence, the individual makes decisions based on information available at the moment which may be imperfect and therefore it led to a situation which is regrettable and not optimal at the moment and costly to change.
    \item Third-degree path Dependence occurs when a decision based on available information leads to not optimal results, even though other alternatives were present, which would lead to better outcome, given the information available at the moment.
\end{itemize}
  
Lock-in occurs when a switching from one vendor, standard or a practice to a another comes with a substantial switching cost, making it inefficient to change and become dependent.

\paragraph{Effect of Path dependency and Lock-in on cloud-based machine learning services}

As for every business, there is a company offering a service or a product and customers on the other side. Therefore, this theory will be applied, and dependencies defined for both sides.

From the Microsoft’s point of view, as the company offering a service, two most important Path dependencies were defined. First arise, when a invest in developing a software solution which become available for customers as a service. It is put on the market, many customers are using it however, it is found out that it is buggy and therefore need costly maintenance. \acrshort{ms} has two options, either keeping the buggy service up for use and paying for maintenance or develop a new service, which will be better but the switching cost higher than current maintenance. 

Another dependency is the hardware one, where the company makes an agreement or partnership with a supplier for the hardware needed to host its services. But if the hardware does not perform to expectations or is too buggy, it will cause the company either a big switching cost to a new hardware or a lot of money for maintenance. 

From the customer’s point of view, a developer may use the service for free to try to develop a machine learning algorithm, but as it becomes more complex, he needs more resources and need to make a decision whether to continue using \acrshort{ms}’s services and pay for them or switch to a competition. At the early phases of development, the switching cost are not high, but if the development team becomes dissatisfied with the service once the solution has been developed, the switching cost are too high. Similarly, it work with bundles that \acrshort{ms} offers with his \acrshort{ml} studio, where the customer may only be interested in using one particular service from the bundle for implementation, he may end up using more of them as they are part of it and becomes lock-in to \acrshort{ms} services. 