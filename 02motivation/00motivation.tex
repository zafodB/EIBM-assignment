\section{Motivation}

There is no need to introduce the concept of cloud computing. The first commercially available cloud was introduced in 2006 by Amazon, with Google following shortly after. Benefits of the cloud computing, such as scalability, affordability and reliability are well known and exploited by countless companies.

Machine learning, on the other hand, was still considered an \textit{emerging technology} in 2017 and was put on top of the Gartner's Hype Cycle~\cite{Panetta2017TopTechnologies}. Since, the hype around machine learning has partially dissolved, but other \acrshort{ai} terms still remain on the curve~\cite{Panetta201852018}. All three major players in the cloud computing market (Amazon, Google and Microsoft) already have tools for \acrshort{ml} in their portfolio.

Most companies in Denmark are still only piloting \acrshort{ai} in their businesses and 73\% is still utilising on-premise hardware as a part of their solution~\cite{EY2018ArtificialDenmark}. It can be expected, that as more and more enterprises will move their \acrshort{ai} to the cloud, the maturity of the cloud based offering will need to increase. The overarching research question could be formulated as follows: \textit{``How can a cloud based machine learning service be offered in a market?''} We explore some parts of this problem by applying theories from the \acrshort{eibm} class.