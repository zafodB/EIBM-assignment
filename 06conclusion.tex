\section{Conclusion}\label{sec:conclusion}

In this Synopsis, four economic theories taught this semester were applied: Business model Canvas, Network effect, Path dependency and Lock-in and Transaction costs.

The \textit{Business Model Canvas} has been filled and all blocks identified to analyse how a big player, such as Microsoft, may be generating revenue and approaching the potential customers. Thanks to this knowledge, the effect of \textit{Network economics} could be explored, identifying the incompatibility of services of biggest players as their biggest weaknesses and strengths. It is also important to know what are the areas, in which the decision made in past, can have a negative impact in the future in order to avoid such decisions and therefore, \textit{Path dependency and Lock-in} theories have been investigated. The last theory looked into was \textit{Transaction cost}, where the Microsoft's current inability to manufacture needed equipment in-house for purposes of his cloud, was identified as one of the biggest transaction costs.

As the technology matures further and the use cases become more clear, we can expect more competition in the field and perhaps even offering of a cloud-based \acrshort{ml} service as a standalone system (as opposed to a bundle offering, tied with other cloud services). It is challenging to objectively evaluate the economic performance of the cloud-based \acrshort{ml} system, as this is currently always bundled with other cloud-based services.